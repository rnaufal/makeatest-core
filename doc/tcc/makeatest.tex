\documentclass{abnt}

\usepackage[utf8]{inputenc}
\usepackage[T1]{fontenc}
\usepackage[brazilian]{babel}
\usepackage[alf]{abntcite}
\usepackage{url}

\ifx\pdftexversion\undefined
\usepackage[dvips]{graphicx}
\else
\usepackage[pdftex]{graphicx}
\usepackage[pdftex,table]{xcolor}
\DeclareGraphicsRule{*}{mps}{*}{}
\fi

\autor{Marcus V. C. Floriano e Debora A. L. Chama}
\titulo{Metadata-based Test Framework to Verify External Effects of the Target Classes}
\orientador{Eduardo M. Guerra}
\instituicao{Aeronautical Institute of Technology \par
Praça Marechal Eduardo Gomes, 50 \par
Vl Acácias - SJ Campos – SP, Brazil}
\local{São Paulo - SP}
\data{03/2012}

\makeindex

\begin{document}

\capa
\folhaderosto

\setlength{\ABNTsignthickness}{1pt}
\begin{folhadeaprovacao}
	Seu texto aqui, onde você deve informar o titulo, data da aprovação (dia, mês e ano), autor do trabalho, local (cidade e estado). E por fim a banca examinadora:
	\assinatura{Prof. fulano \\ Orientador}
	\assinatura{Prof. fulano II \\ Faculdade}
\end{folhadeaprovacao}

\begin{resumo}
	Resumo em português!!!.
\end{resumo}

\begin{abstract}
	Abstract in english!///>>>.
\end{abstract}
\sumario
%corpo do texto do trabalho 
\chapter{Introdução} 
O trabalho tem como principal objetivo apresentar uma ferramenta baseada em metadados para efeitos externos em testes de unidade. Para isso é apresentando conceitos básicos sobre testes de unidade, bem como a relação e problemas enfrentados com dependências externas e as soluções existentes atualmente.\\
Para entendimento completo é apresentado a solução que foi desenvolvida com base em um padrão para frameworks baseados em meta-dados, a extensão do JUnit4 utilizando a funcionalidade "Runners" e a utilização de Proxy Dinâmico como meio de interceptar as chamadas para os métodos dos testes de unidade.\\
\chapter{Conceitos básicos sobre testes de unidade}
\chapter{Utilizando testes de unidade com dependências externas} 
\chapter{Soluções existentes}
\chapter{Framework baseado em meta-dados}
São frameworks cujo processamento lógico é baseado em metadados adicionados nas classes, e as instancias dessas classes trabalham com essa informação. O artigo ``A Pattern Language for Metada-based Frameworks'' \cite{GUERRA-PATTERN} descreve os frameworks que utilizam dessa abordagem, e apresenta um padrão para o desenvolvimento dessas soluções.\\

Basicamente esse padrão é estruturado da seguinte forma:\\

\begin{itemize}
	\item Organization Patterns
		\begin{description}
			\item Esse é a base do padrão, responsável pela organização da estrutura interna do metadados do framework.
		\end{description}
	\item Metadata Reading Patterns
	\item Logic Processing Patterns
\end{itemize}

\chapter{O framework ``Make a Test''}
\section{JUnit 4 - Runners}
\section{Proxy dinâmicos}
\section{Estrutura do ``Make a Test''}
A estrutura está baseada no padrão para frameworks baseads em metadados \cite{GUERRA-PATTERN}, assim temos Organization Patterns, Metada Reading Patterns e Logic Processing Patterns.
\subsection{Organization Patterns}

\subsection{Metada Container}
MakeATestRunner
MakeATestProxy
MetadataReader
MetadataContainer


Primeiro é criado um Descriptor que representa as anotações em valores, Propertie Description e Comparison Descriptor.\\
Um PropertieDescription representa um unico valore, o ComparisonDescriptor representa um conjunto de PropertieDescription.\\

O ComparisonMetadataReader cria o conjunto de valores para o ComparisonDescriptor a partir de uma objeto do tipo CLASS.\\

O ComparisonComponent recupera o ComparisonMetadataReader para recuperar os valores.\\

O Repositorio é utilizado para criar uma unica instancia, que é utilizado o padrão singleton, no Repositorio são criados o ComparisonMetadataReader e um cache que é formado por um hashmap e a chave é a CLASS que contêm as anotações.


\bibliography{makeatest}

\end{document}